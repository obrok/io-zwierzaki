\documentclass[10pt,a4paper]{article}
 
\usepackage[polish]{babel}
\usepackage{polski}
\usepackage[utf8]{inputenc}
\usepackage[T1]{fontenc}
\usepackage{graphicx}
\frenchspacing
 
\title{Inżynieria oprogramowania\\zwierzaki.pl}
\author{Marcin Nowak\\Paweł Obrok\\Paweł Pierzchała}
\date{Kompilacja dokumentu: \today}
 
\begin{document}
\maketitle
\clearpage
 
\tableofcontents
\clearpage
 
\section{Wizja}
 
Celem portalu będzie zrzeszanie oraz usprawnienie akcji podejmowanych przez organizacje oraz wolantariuszy pomagających zwierzętom z całej Polski.
 
\subsection{Zwierzęta}
\subsubsection{Informacje}
Głównym zadaniem portalu będzie katalogowanie bezdomnych zwierząt, pośrednictwo w ich adopacjach, zarówno wirtualnych jak i rzeczywistych, oraz śledzenie ich losów. Na stronie każdego zwierzęcia będą się znajdowały podstawowe informacje, tzn.:
\begin{enumerate}
	\item rodzaj (kot/pies/inne)
	\item dane (rasa, imię, kolor)
	\item czy żyje
	\item jest dzikie, do adopcji lub zaadoptowane
	\item czy zwierzę jest agresywne
	\item obszar, w którym zwierzę aktualnie przebywa
	\item stan zdrowia
	\item informacje o wynikach ewentualnych badań weterynaryjnych i/lub chorobach
	\item zdjęcia
\end{enumerate}
\subsubsection{Kalendarz zdarzeń}
\label{s_kalendarz_zdarzen}
Poza tym każde zwierzę będzie posiadało swój \emph{kalendarz zdarzeń}. Będzie on podzielony na sekcję publiczną i prywatną (do wglądu opiekuna, organizacji i weterynarza). Część publiczna będzie mogła zawierać m.in.:
\begin{enumerate}
	\item orientacyjne daty narodzin/odnalezienia
	\item od kiedy szuka domu
	\item daty wizyt u weterynarza
	\item datę adopcji i rodzaj adopcji
	\item datę ewentualnej zmiany miejsca zamieszkania (np. gdy zwierzę przeprowadzi się razem ze swoim nowym panem)
	\item ewentualną datę śmierci
	\item inne dowolne zdarzenia opisowe
\end{enumerate}
Część prywatna natomiast:
\begin{enumerate}
	\item informacje o wyjeździe do weterynarza
	\item koszt wizyty
	\item wyniki badań
	\item daty przyszłych, umówionych wizyt
	\item inne dowolne zdarzenia opisowe, jak w części publicznej
\end{enumerate}
\subsubsection{Łączenie profili}
Będzie możliwe \emph{łączenie} profili zwierząt, jeśli po jakimś czasie ktoś zorientuje się, że dwa profile zawierają informacje o tym samym zwierzęciu (np. ktoś traci kontakt ze zwierzęciem, ktoś inny odnajduje je miesiąc później i zakłada mu nowy profil).
\subsubsection{Pomoc}
Będzie możliwe oznaczenie profilu zwierzęcia \emph{pilnie potrzebuje pomocy} np. po wypadku lub odnalezieniu w ciężkim stanie.

\subsubsection{Wyszukiwanie/katalogowanie}
Użytkownicy będą mogli wyszukiwać zwierzęta ze względu na wszystkie kryteria podane w profilu. Poza tym będzie dostępny katalog zwierząt (z kategoriami np. psy $\rightarrow$ potrzebujące pomocy).
 
\subsection{Adopcje}
Będą możliwe dwa rodzaje adopcji:
 

\subsubsection{Wirtualne}
Adopcja wirtualna --- polega na zapewnieniu środków finansowych na utrzymanie zwierzęcia, np. operacje na które schronisko nie ma funduszy. Osoba, która \emph{wirtualnie adoptuje} jakiegoś zwierzaka, będzie dostawała aktualne informacje o nim i jego zdjęcia. Wolontariusz lub pracownik schroniska będzie miał możliwość zrobienia kilku/kilkunastu zdjęć i konfiguracji częstości ich wysyłania do użytkownika. Poza tym, wolontariusz lub pracownik schroniska będzie miał możliwość ustawienia \emph{przypominajki}, która będzie notyfikowała o konieczności wysłania zdjęcia lub newsa do użytkownika, który zaadoptował danego zwierzaka.
\subsubsection{Rzeczywiste}
Adopcje rzeczywiste polegające na zabraniu zwierzęcia do domu. Na profilu zwierzaka będzie informacja o tym, kto go zaadoptował (np. nazwa użytkownika) i dane tej osoby (nie publiczne, tylko do wglądu wolontariusza lub organizacji opiekującej się wcześniej danym zwierzakiem).
 
\subsection{Użytkownicy}

\subsubsection{Internauci}
Internauci będą mogli wyszukiwać i przeglądać profile zwierząt. Będą mogli adoptować zwierzęta:
\begin{enumerate}
	\item wirtualnie --- wtedy będą otrzymywali newsy i zdjęcia zwierzaka
	\item rzeczywiście --- wtedy zabiorą zwierzaka do siebie
\end{enumerate}
Poza tym internauci będą mogli przeglądać profile organizacji i weterynarzy. Będą też mieli możliwość zostania wolontariuszami. Mogą też zadawać pytania o zwierzęta oraz zgłaszać informacje o znalezionych zwierzętach.
 
\subsubsection{Wolontariusze}
Użytkownikami portalu będą również wolontariusze. Będą oni mogli dodawać informacje zwierzętach (zarówno odnalezionych jak i znajdujących się w bazie), zdjęcia oraz wydarzenia w kalendarzu. Wszyscy wolonatariusze portalu będą mogli dokonowayć adopcji. Wiele osób opiekujących się zwierzętami nie korzysta z internetu, dlatego wolontariusze powinni mieć możliwość pośrednicznia w takich adopcjach.
 
\subsubsection{Organizacje}
Swoje profile na zwierzaki.pl będą miały organizacje pomagające bezdomnym zwierzętom (np. schroniska). Na stronach organizacji będą znajdowały się m.in.:
\begin{enumerate}
	\item profil działalności wraz z opisem
	\item lokalizacja (adres)
	\item historia przeprowadzanych adopcji
	\item informacje o nadchodzących wydarzeniach (np. zbiórka jedzenia czy zabawek dla schroniska, spotkanie z dyskusją na temat zwierząt, spotkania w szkołach)
	\item aktualności
	\item informacje o tym, w jaki sposób można zostać sponsorem organizacji
	\item informacje o tym, w jaki sposób można przeznaczyć 1\% podatku na rzecz danej organizacji
\end{enumerate}

Pracownicy organizacji będą mogli przydzielać zwierzęta weterynarzom (żeby weterynarze nie musieli sami ich wyszukiwać).
 
\subsubsection{Weterynarze}
Specjalne profile będą mogli zakładać weterynarze. Na swoich stronach będą zamieszczać informacje o przychodni. Będą tam też widoczne zwierzęta, którymi się opiekują (dodane przez wolontariuszy lub organizacje). Tylko weterynarze i upoważnieni przez nich wolontariusze będą mogli edytować status choroby zwierzęcia oraz dodawać informacje o leczeniu do kalendarza (również wydarzenia prywatne).

Weterynarze będą mogli reklamować swoje gabinety banerami.

\subsubsection{Sponsorzy}
Specjalne podstrony będą mogli zakładać sponsorzy. Będą mogli oni opisywać, w jaki sposób pomogli danej organizacji. W zamian za pomoc będą mogli też zamieszczać banery reklamowe w różnych miejscach strony.
 
\subsection{Mapy}
Ze względu na to, że aktywność użytkowników jest ściśle związane z obszarem, w którym się oni znajdują, portal będzie zawierał mapy. Na mapach będą zaznaczone \emph{punkty przyjazne dla zwierzaków} czyli m.in. schroniska, przychodnie weterynaryjne, siedziby organizacji. Poza tym będzie można zaznaczać orientacyjne obszary, w których znajdują się określone zwierzęta.

\section{Funkcjonalność}
\subsection{Aktorzy}
\subsubsection{Internauta}
Osoba przeglądająca portal, niezalogowana. Ma dostęp tylko do publicznych informacji zawartych w systemie.
\subsubsection{Wolontariusz}
Wolontariusz posiada konto w systemie. Może edytować inforamcje związane ze zwierzętami przypisanymi do jego organizacji.
\subsubsection{Organizacja}
Organizacja pomagająca zwierzętom. Może zakładać/usuwać konta wolontariuszy, edytować wszystkie inforamcje o przypisanych do niej zwierzętach. Możliwe jest umieszczenie informacji adresowych/promocyjnych o organizacji.
\subsubsection{Weterynarz}
Weterynarz ma dostęp do informacji o zwierzętach, które przeypisała do niego organizacja, może je także edytować. Weterynarz może umieścić informacje promocyjne o swojej przychodni.
\subsubsection{Sponsor}
Sponsorzy nie mogą edytować danych żadnych zwierzaków. Mogą umieszczać materiały reklamowe oraz opisywać w jaki sposób pomogli danej organizacji.
\subsubsection{Administrator}
Dba o poprawność danych dostarczanych przez organizacje.
\subsection{Przypadki użycia}
\subsubsection{Rejestracja organizacji}
\begin{itemize}
	\item \emph{Aktor główny:} Organizacja
	\item \emph{Zakres:} cały system
	\item \emph{Poziom:} cel użytkownika
	\item \emph{Uczestnicy i interesy:} 
		Organizacja --- zakłada konto w systemie, aby móc z niego korzystać\\
		Administrator --- weryfikuje informacje podane przez Organizację
	\item \emph{Warunek początkowy:} brak
	\item \emph{Minimalna gwarancja:} Informacja dla Administratora o prośbie o założenie konta
	\item \emph{Gwarancja powodzenia:} Założenie aktywnego konta w systemie
	\item \emph{Scenariusz}
	\begin{enumerate}
		\item Organizacja podaje informacje teleadresowe, zakres działalności
		\item Administrator weryfikuje informacje
		\begin{enumerate}
			\item Informacje są niepoprawne
			\item Organizacja otrzymuje prośbę o uzupełnienie informacji
		\end{enumerate}
		\item Organizacja otrzymuje konto w systemie
	\end{enumerate}
\end{itemize}

\subsubsection{Rejestracja użytkownika}
\begin{itemize}
	\item \emph{Aktor główny:} Organizacja
	\item \emph{Zakres:} organizacja
	\item \emph{Poziom:} cel użytkownika
	\item \emph{Uczestnicy i interesy:} \\
		Organizacja --- zakłada konto dla Wolontariusza/Weterynarza/Sponsora w systemie\\
		Wolontariusz/Weterynarz/Sponsor --- otrzymuje konto w systemie
	\item \emph{Warunek początkowy:} posiadanie konta w systemie przez Organizację
	\item \emph{Minimalna gwarancja:} brak
	\item \emph{Gwarancja powodzenia:} Założenie konta w systemie
\end{itemize}

\subsubsection{Logowanie}
\begin{itemize}
	\item \emph{Aktor główny:} Organizacja/Wolontariusz/Weterynarz/Sponsor
	\item \emph{Zakres:} organizacja
	\item \emph{Poziom:} cel użytkownika
	\item \emph{Uczestnicy i interesy:} \\
		Wolontariusz/Weterynarz/Sponsor/Organizacja --- potwierdza swoje uprawnienia do działań w systemie
	\item \emph{Warunek początkowy:} posiadanie konta w systemie przez Organizację/ Wolontariusza/ Weterynarza/ Sponsora
	\item \emph{Minimalna gwarancja:} informacja o niepoprawnych danych logowania
	\item \emph{Gwarancja powodzenia:} autoryzacja w systemie
\end{itemize}

\subsubsection{CRUD Zwierzaków}
\begin{itemize}
	\item \emph{Aktor główny:} Organizacja/Wolontariusz/Weterynarz
	\item \emph{Zakres:} organizacja
	\item \emph{Poziom:} cel użytkownika
	\item \emph{Uczestnicy i interesy:} 
		Organizacja/Wolontariusz/Weterynarz --- dodaje aktualne informacje o zwierzaku
	\item \emph{Warunek początkowy:} użytkownik jest zaautoryzowany
	\item \emph{Minimalna gwarancja:} informacja o błędzie
	\item \emph{Gwarancja powodzenia:} uaktualnienie informacji o zwierzęciu
	\item \emph{Scenariusz}
	\begin{enumerate}
		\item Użytkownik podaje aktualne informacje o zwierzęciu (tyczy się to również dodawania zwierząt).
		\item Weryfikowane są uprawnienia Użytkownika do wykonania danej akcji
		\begin{enumerate}
			\item Organizacja może dodawać informacje o nowych zwierzętach, aktualizować dane dowolnego zwierzęcia przypisanego do niej
			\item Wolontariusz może zmieniać tylko aktualizować dane zwierzęcia
			\item Weterynarz może zmieniać informacje o stanie zdrowia zwierzęcia
			\item Nie ma możliwości usuwania zwierzaka z systemu. Co najwyżej można dodać zdarzenie śmierci zwierzaka.
		\end{enumerate}
		\item Informacje są niepoprawne
		\begin{enumerate}
			\item Użytkownik widzi komunikat o niepowodzeniu i jego przyczynie.
			\item Użytkownik może ponownie spróbować zmienić informacje.
		\end{enumerate}
		\item Dane zwierzaka są aktualizowane (zwierzak jest dodawany do systemu).
	\end{enumerate}
\end{itemize}

\subsubsection{Dodawanie zdarzenia}
\begin{itemize}
	\item \emph{Aktor główny:} Wolontariusz/Weterynarz
	\item \emph{Zakres:} zwierzę
	\item \emph{Poziom:} cel użytkownika
	\item \emph{Uczestnicy i interesy:} 
		Wolontariusz/Weterynarz --- dodaje informacje o historii zwierzęcia
	\item \emph{Warunek początkowy:} użytkownik jest zaautoryzowany
	\item \emph{Minimalna gwarancja:} informacja o błędzie
	\item \emph{Gwarancja powodzenia:} uaktualnienie informacji o historii zwierzęcia
	\item \emph{Scenariusz}
	\begin{enumerate}
		\item Użytkownik wybiera rodzaj zdarzenia (patrz \ref{s_kalendarz_zdarzen}).
		\item Użytkownik dostarcza ewentualnie inne informacje związane ze zdarzeniem.
		\item Możliwe jest wskazanie lokalizacji wydarzenia na mapie.
		\item Użytkownik ustala widoczność zdarzenia (prywatne, publiczne)
		\item Zdarzenie jest dodawane do historii zwierzaka.
	\end{enumerate}
\end{itemize}

\subsubsection{Wyszukiwanie}
\label{s_wyszukiwanie}
\begin{itemize}
	\item \emph{Aktor główny:} Internauta
	\item \emph{Zakres:} cały system
	\item \emph{Poziom:} cel użytkownika
	\item \emph{Uczestnicy i interesy:} 
		Internauta --- wyszukuje zwierzaka
	\item \emph{Warunek początkowy:} brak
	\item \emph{Minimalna gwarancja:} informacja o braku zwierząt odpowiadających danym kryteriom
	\item \emph{Gwarancja powodzenia:} wyświetlenie listy zwierzą odpowiadających danym kryteriom
	\item \emph{Scenariusz}
	\begin{enumerate}
		\item Użytkownik podaje kryteria wyszukiwania.\\
		Możliwe kryteria wyszukiwania
		\begin{enumerate}
			\item Rasa
			\item Wiek
			\item Płeć
			\item Lokalizacja
			\item Możliwość adpocji
			\item Umaszczenie
		\end{enumerate}	
		\item Wyświetlana jest lista zwierząt odpowiadających podanym kryteriom
		\begin{enumerate}
			\item Brak zwierząt odpowiadających kryteriom
			\item Użytkownik widzi komunikat o niepowodzeniu
		\end{enumerate}
	\end{enumerate}
\end{itemize}


\subsubsection{Adopcja wirtualna}
\begin{itemize}
	\item \emph{Aktor główny:} Internauta
	\item \emph{Zakres:} zwierzę
	\item \emph{Poziom:} cel użytkownika
	\item \emph{Uczestnicy i interesy:} 
		Internauta --- zgadza się sponsorować zwierzaka
	\item \emph{Warunek początkowy:} brak
	\item \emph{Minimalna gwarancja:} brak
	\item \emph{Gwarancja powodzenia:} dodanie informacji o adpocji wraz z danymi kontaktowymi osoby adoptującej
	\item \emph{Scenariusz}
	\begin{enumerate}
		\item Użytkownik wyszukuje (\ref{s_wyszukiwanie}) odpowiadającego mu zwierzaka
		\item Użytkownik zgłasza chęć wirtualnej adpocji
		\item Organizacja przekazuje użytkownikowi niezbędne dokumenty oraz numer konta, na który należy dokonywać wpłat
		\item Po potwierdzeniu płatności przez użytkownika dopisywany jest on jako opiekun zwierzęcia
	\end{enumerate}
\end{itemize}

\subsubsection{Adopcja}
\begin{itemize}
	\item \emph{Aktor główny:} Internauta
	\item \emph{Zakres:} zwierzę
	\item \emph{Poziom:} cel użytkownika
	\item \emph{Uczestnicy i interesy:} 
		Internauta --- chce przygarnąć zwierzaka
	\item \emph{Warunek początkowy:} brak
	\item \emph{Minimalna gwarancja:} brak
	\item \emph{Gwarancja powodzenia:} dodanie informacji o adpocji wraz z danymi kontaktowymi osoby adoptującej
	\item \emph{Scenariusz}
	\begin{enumerate}
		\item Użytkownik wyszukuje (\ref{s_wyszukiwanie}) odpowiadającego mu zwierzaka
		\item Użytkownik zgłasza chęć adpocji
		\item Organizacja przekazuje użytkownikowi niezbędne dokumenty oraz informacje o miejscu odbioru zwierzaka
		\item Po odebraniu zwierzęcia dodawane są informacje adresowe adoptującego na potrzeby przyszłych kontroli prze Ministerstwo Ochrony Biednych Zwierząt.
	\end{enumerate}
\end{itemize}

\subsubsection{Przeglądanie mapy}
\begin{itemize}
	\item \emph{Aktor główny:} Internauta
	\item \emph{Zakres:} cały system
	\item \emph{Poziom:} cel użytkownika
	\item \emph{Uczestnicy i interesy:} 
		Internauta --- przegląda informacje związane z pewnym rejonem
	\item \emph{Warunek początkowy:} brak
	\item \emph{Minimalna gwarancja:} wyświetlenie mapy rejonu
	\item \emph{Gwarancja powodzenia:} wyświetlenie mapy rejonu wraz z informacjami o zwierzętach i zdarzeniach
\end{itemize}

\subsection{Testy akceptacyjne}
Testy akceptacyjne systemu powinny odtwarzać wyżej opisane przypadki użycia. Funkcjonalość systemu nie jest bardzo rozbudowana i scenariusze przypadków użycia opisują ją dość dokładnie.

\end{document}